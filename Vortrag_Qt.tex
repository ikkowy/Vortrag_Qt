\documentclass[aspectratio=169]{beamer}

\usetheme{nutria}

\usepackage[ngerman]{babel}

\usepackage{fontspec}
\usepackage{unicode-math}
\setmonofont{Inconsolata}
\setmathfont{TeX Gyre Pagella Math}

\usepackage{pifont}

\usepackage{listings}

\lstset{
basicstyle=\ttfamily
}

\usepackage{mdframed}
\usepackage{graphicx}

%-------------------------------------------------------------------------------

\title{Anwendungsentwicklung in C++ unter Verwendung des Qt Frameworks}
\author{Lan Nguyen \and Vincent Westenberg}
\institute{Berufsakademie Sachsen -- Staatliche Studienakademie Leipzig}
\date{12. November 2020}

%-------------------------------------------------------------------------------
\begin{document}
%-------------------------------------------------------------------------------

\begin{frame}
\titlepage
\end{frame}

%-------------------------------------------------------------------------------

\begin{frame}{Inhalt}
\begin{enumerate}
\item Was ist C++?
\item Was ist Qt?
\item Beispiele
\end{enumerate}
\end{frame}

%-------------------------------------------------------------------------------

\begin{frame}{Was ist C++?}
\begin{itemize}
\item eine zu C kompatible Programmiersprache
\item[\ding{237}] C entworfen 1972 von Dennis Ritchie für das Unix Betriebssystem
\item C++ entworfen 1985 von Bjarne Stroustrup
\item solider Standard (C++ 98/11/14/17/20)
\item[\ding{237}] C-Programme allgemein problemlos mit C++ Compiler kompilierbar
\item performant, systemnah, objektorientiert, funktional
\item[\ding{237}] inline-Assembler, Erben von mehreren Klassen, Überladen von Operatoren, Lambda-Ausdrücke
\item kein Garbage-Collector (echtzeitfähig)
\item[\ding{237}] manuelle Speicherverwaltung, Referenzzählung
\item[\ding{237}] Programmierung von Mikrocontrollern/Echtzeitsystemen
\end{itemize}
\end{frame}

%-------------------------------------------------------------------------------

\begin{frame}[fragile]{Beispiel \texttt{hello.cpp}}
\begin{lstlisting}[language=C++]
#include <iostream>

int main () {
    std::cout << "Hello world!" << std::endl;
    return 0;
}
\end{lstlisting}
\end{frame}

%-------------------------------------------------------------------------------

\begin{frame}[fragile]{Beispiel \texttt{sort.cpp}}
\scriptsize
\vspace*{-5pt}
\begin{lstlisting}[language=C++]
#include <iostream>
#include <algorithm>
#include <cstdlib>
#include <vector>

void dump (const std::vector<int> & numbers) {
    for (auto & n : numbers) {
	std::cout << n << (&n == &numbers.back() ? '\n' : ' ');
    }
}

int main () {
    std::vector<int> numbers;
    srand(time(NULL));
    for (int i=0; i < 10; i++) { numbers.push_back(rand() % 100); }
    dump(numbers);
    std::sort(numbers.begin(), numbers.end());
    dump(numbers);
    return 0;
}
\end{lstlisting}
\end{frame}

%-------------------------------------------------------------------------------

\begin{frame}[fragile]{Was ist Qt?}
\begin{flushright}
\includegraphics[scale=0.1]{logo}
\end{flushright}
\begin{itemize}
\item umfangreiche C++ Bibliothek
\item plattformübergreifend
\item GUI-Toolkit
\item Open Source
\item Anbindung an diverse Programmiersprachen
\end{itemize}
\end{frame}

%-------------------------------------------------------------------------------

\begin{frame}{Geschichte und Entwicklung}
\begin{itemize}
\item 1991: entwickelt von Haavard Nord und Eirik Chambe-Eng
\item 1993: Gründung von Quasar Technologies (später Trolltech)
\item 1996: Qt 1.0
\item 2008: von Nokia aufgekauft
\item 2011: unter freie Lizenz gestellt (duales Lizenzmodell)
\item heute: Qt 5.15
\end{itemize}
\end{frame}

%-------------------------------------------------------------------------------

\begin{frame}{Funktionsumfang}
\begin{itemize}
\item Unterteilung in Module
\item[\ding{237}] Core, GUI, Quick, Multimedia, Network, SQL, \ldots
\item Anbindung an Python
\item Qt Quick nutzt QML und JavaScript
\item Datenbankzugriff, XML, JSON
\end{itemize}
\end{frame}

%-------------------------------------------------------------------------------

\begin{frame}{Signals und Slots}
\begin{itemize}
\item Qt erweitert die Programmiersprache C++ um ein Signal-Slot-System
\item[\ding{237}] MOC (Meta Object Compiler)
\item[\ding{237}] Signals: Ereignisse, zum Beispiel Klick auf ein Objekt
\item[\ding{237}] Slots: definierte Aktionen für das Eintreten eines Signals
\end{itemize}
\end{frame}

%-------------------------------------------------------------------------------

\begin{frame}{Beispiel Julia-Mengen}
\begin{center}
{\Large \(f : \mathbb{C} \rightarrow \mathbb{C}, \quad z \mapsto z^2+c \quad (c \in \mathbb{C})\)}
\end{center}

\[ z^2 ~=~ (a+bi)^2 ~=~ (a+bi)(a+bi) ~=~ \underbrace{a^2 - b^2}_\Re + \underbrace{2abi}_\Im \]
\end{frame}

%-------------------------------------------------------------------------------

\begin{frame}{Quellen}
\begin{itemize}
\item \url{https://doc.qt.io/}
\item Jürgen Wolf. \textit{C++ von A bis Z}. Galileo Computing. 2006
\item \url{https://wiki.qt.io/Qt_History}
\item \url{https://www.qt.io/licensing/}
\end{itemize}
\end{frame}

%-------------------------------------------------------------------------------

\begin{frame}
\begin{center}
\includegraphics[scale=4]{huhn}

\vspace*{30pt}
{\Huge \textbf{Vielen Dank für Ihre Aufmerksamkeit}}
\end{center}
\end{frame}

%-------------------------------------------------------------------------------
\end{document}
%-------------------------------------------------------------------------------
